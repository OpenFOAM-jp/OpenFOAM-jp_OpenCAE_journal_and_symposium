\documentclass{ltjoc}
%
\usepackage{subfigure}
\usepackage{graphicx}
\usepackage[
 pdfencoding=auto,
 bookmarks=true,
 bookmarksnumbered=true,
 colorlinks=true,
 allcolors={blue}
]{hyperref}
\renewcommand*{\figureautorefname}{図}
\renewcommand*{\tableautorefname}{表}
% フォント
\usepackage{luatexja-otf}
\usepackage[ms]{luatexja-preset} % 和文フォント: MSフォント
%\usepackage[ipa]{luatexja-preset} % 和文フォント: IPAフォント
%\usepackage[hiragino-pro]{luatexja-preset} % 和文フォント: ヒラギノフォント
\setmainfont[Ligatures=TeX]{Cambria} % 欧文通常フォント: Cambria
\setsansfont[Ligatures=TeX]{Calibri} % 欧文Sans-serifフォント: Calibri
%
\markright{OpenCAE Symposium 2018, Dec. 8-9, 2018, Kawasaki}
\title{オープンCAEシンポジウム2018印刷用原稿の書き方(和文表題)}
\subtitle{\TeX のテンプレート(和文副題)}
\author{%
氏 名$^{1\dagger}$%
\hspace{1\zw}%
氏 名$^{2}$%
\hspace{1\zw}%
氏 名$^{3}$%
}
\affiliation{%
${}^{1}$所属%
\hspace{1\zw}%
${}^{2}$所属%
\hspace{1\zw}%
${}^{3}$所属%
} 
\etitle{Manuscripts preparation guide for the OpenCAE symposium 2018 (English Title)}
\esubtitle{The case of \TeX (English Sub-Title)}
\eauthor{%
Firstname FAMILYNAME$^{*}$%
\hspace{1em}%
Firstname FAMILYNAME$^{**}$%
\hspace{1em}%
Firstname FAMILYNAME$^{***}$%
}
\eaffiliation{%
${}^{*}$affiliation%
\hspace{1em}%
${}^{**}$affiliation%
\hspace{1em}%
${}^{***}$affiliation%
}
\keywords{Keyword1, Keyword2, Keyword3, Keyword4, Keyword5}
\AuthorsEmail{corresponding.author@opencae.or.jp}
\begin{abstract}
Abstract Abstract Abstract Abstract Abstract
Abstract Abstract Abstract Abstract Abstract 
Abstract Abstract Abstract Abstract Abstract
Abstract Abstract Abstract Abstract Abstract 
\end{abstract}
%
\begin{document}
\maketitle
%%%
\section{原稿について}
%%
\subsection{様式}
\begin{itemize}
\item 用紙 : A4です.
\item 余白 : 上下25mm,左右20mmです.
  英文概要とキーワードの行は,さらに左右6.5mmの余白を設けてください.
\item 段組 : 1段組です.
\item 頁数 : 最小1ページ,最大10ページ,標準2〜4ページとします.
\end{itemize}
%%
\subsection{本文}
\begin{itemize}
\item 言語 : 日本語または英語です.
\item 句読点 : 本文が日本語の場合,句読点として,
  全角の読点「,」(カンマ)と句点「.」(ピリオド)を用いてください. 
\item 段落最初の行の字下げ : 1文字です.
\item 題目・所属
  \begin{itemize}
  \item 
    本文が日本語の場合,日本語での題目,著者名と所属の記載に続けて,
    英語でも同内容を記載してください.
  \item 
    本文が英語の場合,英語の題目および著者名と所属のみの記載でも結構です.
  \end{itemize}
\item 英文概要 : 40から80ワード程度で記載ください.
  なお,英文概要は省略しても構いません.
\item キーワード : 講演内容を良く表すキーワードを最低3語,通常5語程度選定し,英語で記入ください.
\item 字体
  \begin{itemize}
  \item 本文 :
    明朝体・Serif系(Cambriaなど)を使用してください.
  \item 題目・著者名・所属・見出し・図表キャプション : 
    ゴシック体・Sans-serif系(Calibriなど)を使用してください.
    キーワードの見出し(Keywords)はSans-serifのイタリック体を使用してください.
  \end{itemize}
\item 文字の大きさ
  \begin{itemize}
  \item 題目 : 14ポイント
  \item 副題・大見出し : 12ポイント
  \item ヘッダ・フッタ : 8ポイント
  \item 上記以外(本文・著者名・所属・Abstract・キーワード・
      図表キャプション・大見出し以外の見出し・ページ番号など) : 10ポイント
  \end{itemize}
\item 参考文献
  \begin{itemize}
  \item 
  参考文献は,本文中の引用箇所の末尾に角括弧をつけた番号で表し,
  本文の末尾にまとめて列記してください.
  \item 
  WEBページについても,
  参考文献\cite{SIST02-200}等を参考にして,
  URLやアクセス日付を明記してください.
  \end{itemize}
\end{itemize}
%%
\subsection{図表}
\begin{itemize}
\item 言語 : 図表中の記号およびキャプションは英語とします.
\item 字体・文字の大きさ : 
  明瞭である限り,図表中の字体や文字の大きさは任意です.
\item 本文中での参照 : \autoref{fig:Logo},
  \autoref{tab:Secretariat}などと記載してください.
\end{itemize}
%
\begin{figure}[htbp]
\centering
\subfigure[Color logo]{
\includegraphics[width=0.2\textwidth]{fig/Logo-c3-1.jpg}
\label{fig:Logo-c3-1}
}
\hspace{0.1\textwidth}
\subfigure[Monochrome logo]{
\includegraphics[width=0.2\textwidth]{fig/Logo-b3.jpg}
\label{fig:Logo-b3}
}
\caption{Logo of the OpenCAE Society Japan}
\label{fig:Logo}
\end{figure}
%%%
\section{原稿提出について}
%%
\begin{itemize}
\item ファイル形式 : PDF形式で提出ください.
\item ファイルサイズ : 最大20MBです.
\item フォント : 提出されるPDFファイルに全てのフォントが埋め込れている事を確認ください.
\item URL : リンクを付加してください.
  また,可能であれば,図表の参照についてもリンクを付加してください.
  リンクされた文字は青色としてください.
\item 提出方法 : オープンCAEシンポジウム2018のWEBページ
  \href
  {http://www.opencae.or.jp/activity/symposium/opencae_symposium2018/}
  {http://www.opencae.or.jp/activity/symposium/opencae\_symposium2018/}
  をご参照の上,提出してください.
\end{itemize}
%%%
\section{お問い合わせ}
ご不明点については,
お手数ですが,
\autoref{tab:Secretariat}のシンポジウム事務局まで,
電子メールでお問い合わせください.
\begin{table}[htbp]
  \centering
  \begin{tabular}{l|l}
    \hline
    E-mail address & symposium2018@opencae.or.jp\\
    \hline
    \end{tabular}
  \caption{Secretariat of the OpenCAE symposium 2018}
  \label{tab:Secretariat}
\end{table}
%
\begin{thebibliography}{9}
\bibitem{SIST02-200}
科学技術情報流通技術基準
参照文献の書き方
SIST 02 – 2007,
独立行政法人科学技術振興機構(2007).
\href
{http://jipsti.jst.go.jp/sist/pdf/SIST02-2007.pdf}
{http://jipsti.jst.go.jp/sist/pdf/SIST02-2007.pdf}
(accessed 2015-09-29).
\end{thebibliography}
%
\end{document}
