\documentclass{ltjoc}
%
\usepackage{subfigure}
\usepackage{graphicx}
\usepackage[
 pdfencoding=auto,
 bookmarks=true,
 bookmarksnumbered=true,
 colorlinks=true,
 allcolors={blue}
]{hyperref}
\renewcommand*{\figureautorefname}{図}
\renewcommand*{\tableautorefname}{表}
% フォント
\usepackage{luatexja-otf}
\usepackage[ms]{luatexja-preset} % MSフォント
%\usepackage[ipa]{luatexja-preset} % IPAフォント
%\usepackage[hiragino-pro]{luatexja-preset} % ヒラギノフォント
%
\markright{OpenCAE Symposium 2018, Dec. 8-9, 2018, Kawasaki}
\title{オープンCAEシンポジウム2018印刷用原稿の書き方(和文表題)}
\subtitle{\LaTeX のテンプレート(和文副題)}
\author{%
氏 名$^{1\dagger}$%
\hspace{1\zw}%
氏 名$^{2}$%
\hspace{1\zw}%
氏 名$^{3}$%
}
\affiliation{%
${}^{1}$所属%
\hspace{1\zw}%
${}^{2}$所属%
\hspace{1\zw}%
${}^{3}$所属%
} 
\etitle{Manuscripts preparation guide for the OpenCAE symposium 2018 (English Title)}
\esubtitle{The case of \LaTeX (English Sub-Title)}
\eauthor{%
Firstname FAMILYNAME$^{*}$%
\hspace{1em}%
Firstname FAMILYNAME$^{**}$%
\hspace{1em}%
Firstname FAMILYNAME$^{***}$%
}
\eaffiliation{%
${}^{*}$affiliation%
\hspace{1em}%
${}^{**}$affiliation%
\hspace{1em}%
${}^{***}$affiliation%
}
\keywords{Keyword1, Keyword2, Keyword3, Keyword4, Keyword5}
\AuthorsEmail{corresponding.author@opencae.or.jp}
\begin{abstract}
Abstract Abstract Abstract Abstract Abstract Abstract Abstract Abstract Abstract Abstract Abst.
\end{abstract}
%
\begin{document}
\maketitle
%%%
\section{原稿について}
%%
\subsection{様式}
\begin{description}
\item [用紙]
  A4です.
\item [余白]
  上25mm,下25mm,右20mm,左20mmです.
\item [段組]
  1段組です.
\item [頁数]
  最小1ページ,最大{\bfseries 10ページ}です.
\end{description}
%%
\subsection{本文}
\begin{description}
\item [言語]
  日本語または英語です.
\item [句読点]
  本文が日本語の場合,
  句読点として,
  全角の読点「,」(カンマ)と句点「.」(ピリオド)を用いてください. 
\item [題目・所属]
  題目と所属については,以下とします.
  \begin{itemize}
  \item 
  本文が日本語の場合,日本語での題目,著者名と所属の記載に続けて,
  英語でも同内容を記載してください.
  \item 
  本文が英語の場合,英語の題目および著者名と所属のみの記載でも結構です.
  \end{itemize}
\item [概要]
  英文概要を40から80ワード程度で記載ください.
  なお,英文概要は省略しても構いません.
\item [キーワード]
  講演内容を良く表すキーワードを最低3語,通常5語程度選定し,英語で記入ください.
\item [字体]
  字体は,それぞれ以下とします.
  \begin{description}
  \item 
    [本文]
    明朝体・Serif系(Times New Romanなど)を使用してください.
  \item 
    [題目・所属・キーワード・見出し]
    ゴシック体・Sans-serif系(Helveticaなど)を使用してください.
  \end{description}
\item [文字の大きさ]
  フォントの大きさは,それぞれ以下とします.
  \begin{description}
  \item [題目]
    14ポイント
  \item [副題・大見出し]
    12ポイント
  \item [本文・所属・概要・Abstract・キーワード・図表キャプション・大見出し以外の見出し]
    10ポイント
  \end{description}
\item [参考文献]
  参考文献については,以下とします.
  \begin{itemize}
  \item 
  参考文献は,本文中の引用箇所の末尾に角括弧をつけた番号で表し,
  本文の末尾にまとめて列記してください.
  \item 
  WEBページについても,
  参考文献\cite{SIST02-200}等を参考にして,
  URLやアクセス日付を明記してください.
  \end{itemize}
\end{description}
%%
\subsection{図表}
\begin{description}
\item [言語]
  図表中の記号およびキャプションは英語とします.
\item [字体・文字の大きさ]
  明瞭である限り,図表中の字体や文字の大きさは任意です.
\item [本文中での参照]
  \autoref{fig:Logo},
  \autoref{tab:Secretariat}などと記載してください.
\end{description}
%
\begin{figure}[htbp]
\centering
\subfigure[Color logo]{
\includegraphics[width=0.2\textwidth]{fig/Logo-c3-1.jpg}
\label{fig:Logo-c3-1}
}
\hspace{0.1\textwidth}
\subfigure[Monochrome logo]{
\includegraphics[width=0.2\textwidth]{fig/Logo-b3.jpg}
\label{fig:Logo-b3}
}
\caption{Logo of the OpenCAE Society Japan}
\label{fig:Logo}
\end{figure}
%%%
\section{原稿提出について}
%%
\begin{description}
\item [ファイル形式]
  PDF形式で提出ください.
\item [ファイルサイズ]
  最大{\bfseries 20MB}です.
\item [フォント]
  提出されるPDFファイルに全てのフォントが埋め込れている事を確認ください.
\item [URL]
  URLにはリンクを付加してください.
  また,可能であれば,図表の参照についてもリンクを付加してください.
  なお,リンクされた文字は青色としてください.
\item [提出方法]
  オープンCAEシンポジウム2018のWEBページ
  \href
  {http://www.opencae.or.jp/activity/symposium/opencae_symposium2018/}
  {http://www.opencae.or.jp/activity/symposium/opencae\_symposium2018/}
  をご参照の上,提出してください.
\end{description}
%%%
\section{お問い合わせ}
ご不明点については,
お手数ですが,
\autoref{tab:Secretariat}のシンポジウム事務局まで,
電子メールでお問い合わせください.
\begin{table}[htbp]
  \centering
  \begin{tabular}{l|l}
    \hline
    E-mail address & symposium2018@opencae.or.jp\\
    \hline
    \end{tabular}
  \caption{Secretariat of the OpenCAE symposium 2018}
  \label{tab:Secretariat}
\end{table}
%
\begin{thebibliography}{9}
\bibitem{SIST02-200}
科学技術情報流通技術基準
参照文献の書き方
SIST 02 – 2007,
独立行政法人科学技術振興機構(2007).
\href
{http://jipsti.jst.go.jp/sist/pdf/SIST02-2007.pdf}
{http://jipsti.jst.go.jp/sist/pdf/SIST02-2007.pdf}
(accessed 2015-09-29).
\end{thebibliography}
%
\end{document}
